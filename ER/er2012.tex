
% ER 2012: International Conference on Conceptual Modeling

\documentclass[runningheads,a4paper]{llncs}

\usepackage{amssymb}
\setcounter{tocdepth}{3}
\usepackage{graphicx}

\usepackage[latin1]{inputenc}

\usepackage{url}
\urldef{\mailsa}\path|{alfred.hofmann, ursula.barth, ingrid.haas, frank.holzwarth,|
\urldef{\mailsb}\path|anna.kramer, leonie.kunz, christine.reiss, nicole.sator,|
\urldef{\mailsc}\path|erika.siebert-cole, peter.strasser, lncs}@springer.com|    
\newcommand{\keywords}[1]{\par\addvspace\baselineskip
\noindent\keywordname\enspace\ignorespaces#1}

\begin{document}

\mainmatter  % start of an individual contribution

\title{The MOSKitt4ME Approach: Providing Process Support in a Method
Engineering Context\thanks{This work has been developed with the support of
MICINN under the project EVERYWARE TIN2010-18011.}}

\author{Mario Cervera \and Manoli Albert \and Victoria Torres \and Vicente
Pelechano}

\institute{Centro de Investigaci\'on en M\'etodos de Producci\'on de Software\\ 
Universidad Polit\'ecnica de Valencia \\ Camino de Vera s/n, 46022 Valencia,
Spain\\
\email{\{mcervera,malbert,vtorres,pele\}@pros.upv.es}
}

\maketitle

\begin{abstract}

It is commonly agreed that software developments methods must be defined
(or adapted) in-house in order to meet the particular needs of the organizations
where they are to be applied. To help meet this challenge, Method Engineering
(ME) research aims to provide solutions to efficiently deal with the definition
and adaptation of methods, and the construction of the supporting software
tools. However, while the product part of methods is fully considered by most ME
approaches, the specification and enactment of the process part is less
well-supported. To fill this gap, this work presents a methodological ME
approach and a Computer-Aided Method Engineering (CAME) environment (MOSKitt4ME)
that support the design and implementation of the process part of methods in the
context of Model-Driven Engineering. The proposal is illustrated by means of a
real case study from the Valencian Regional Ministry of Infrastructure,
Territory and Environment.

\keywords{Method Engineering, CAME Environment, Process Support, Model-Driven
Engineering}
\end{abstract}

\section{Introduction}

The definition of a software development method suitable for all situations is
now considered unfeasible \cite{Cockburn00,HendersonSellers10}. For
this reason, software organizations need to define (or adapt) their methods
in-house in order to meet their specific needs. To help meet this challenge,
Method Engineering (ME) research aims to provide solutions
\cite{Brinkkemper99,Karlsson04,Prakash97,Ralyte01a} to efficiently deal with the
definition and adaptation of methods and also with the construction of the
supporting software tools.

Similarly to software engineering, which is concerned with all aspects of
software and its development, ME is concerned with all aspects of methods and
their definition. Thus, most ME approaches define precise engineering solutions
that address the definition of the two interrelated aspects that generally
comprise methods: product and process \cite{Brinkkemper99,Prakash97,Ralyte01a}.
However, while it is commonly agreed that the product aspect of methods
represents the artifacts to be produced during the method execution, the process
aspect is usually understood in two slightly different ways. Some consider the
process aspect as the overall development process of the method, which
encompasses all the activity-related issues needed for software development
\cite{Niknafs09}. By contrast, most ME approaches (e.g.,
\cite{Brinkkemper99,Karlsson09,Prakash99,Ralyte01a}) use the term process at a
smaller scale, considering a process as the description of how a single method
product must be built.

In this work, we consider processes at the greater scale (hence, we denote
hereafter the overall process of methods simply as the method \textit{process
part}\footnote{Likewise, we denote the product aspect of methods as the method
\textit{product part}.}). We argue that supporting the specification and the
enactment of the process part of methods brings important benefits. A
precise, complete, and well-structured process specification may be useful to
facilitate the understanding of how software development is performed within an
organization. Furthermore, the enactment of this process specification in a
software environment may be useful to guide software engineers throughout the
actual development process and also to automate it as far as possible. However,
to the best of our knowledge there is no ME approach that supports the
specification and the enactment of the process part of methods, and also
provides complete software support to these issues.

In order to fill this gap, this paper describes a methodological approach and a
supporting software environment that support the specification of the process
part of methods and also the construction of the software tools required to
enact this process part. The proof of concept is performed in the context of a
ME approach and a Computer-Aided Method Engineering (CAME) environment, called
MOSKitt4ME\footnote{MOSKitt4ME is an extension of MOSKitt
(http://www.moskitt.org/) for ME.}, that have been presented by the authors in
\cite{Cervera10,Cervera11}. This ME approach is based on Model-Driven
Engineering (MDE) techniques and thereby it proposes defining methods as method
models based on the SPEM 2.0 standard \cite{spem} (\textit{method design} phase)
and semi-automatically building the supporting CASE environments by means of
model transformations (\textit{method implementation} phase). This paper focuses
on extending both the ME proposal and the MOSKitt4ME tool to enable process
specification (during method design) and process enactment (during method
implementation). Specifically, the proposal and the CAME environment have been
enhanced with an executable process modeling language (BPMN 2.0 \cite{bpmn}) to
properly support the specification of the process part of methods. Moreover, the
proposal and the CAME environment have also been extended to support the
generation of CASE environments that incorporate process enactment through the
use of a process engine.

The proposal and the CAME environment presented in this paper are being used at
the Valencian Regional Ministry of Infrastructure, Territory and Environment,
also known as CIT. Specifically, the gvMetrica method has been specified using
MOSKitt4ME and a CASE environment has been generated from this specification.
Moreover, gvMetrica is currently being executed in real projects using this CASE
environment. The application of the proposal in a real context is allowing us to
take initial feedback that can be used to improve it.

The paper is structured as follows. First, section \ref{SectionStateOfTheArt}
presents a brief state-of-the-art review that highlights some of the process
support limitations that present current ME approaches. Then, section
\ref{SectionOverview} provides an overview of our proposal. Section
\ref{SectionMethodDesign} describes how the process specification is performed
during the method design. Section \ref{SectionMethodImplementation} describes
how the process enactment is supported in the method implementation. Finally,
section \ref{SectionConclusions} draws some conclusions.

\section{State of the Art}
\label{SectionStateOfTheArt}

The term Method Engineering was introduced in the mid-eighties by Bergstra
\textit{et al}. in \cite{Bergstra85}. Thereafter, many research efforts have
attempted to provide solutions to the challenges that ME entails. Some of the
most relevant contributions are those by Brinkkemper
\cite{Brinkkemper96,Brinkkemper99}, Prakash
\cite{Prakash97,Prakash99}, Ralyt\'e \cite{Ralyte01a,Ralyte01b} and Karlsson
\cite{Karlsson04,Karlsson09}. These works are based on a modular view of
methods, whereby methods are built by assembling different kinds of methods
modules, namely method fragments, method blocks, method chunks and method
components respectively. With regard to the process support, these modules focus
on the processes necessary to develop specific method products. The method
process part is defined when these modules are assembled, generally by means of
precedence relationships that establish their execution order. Thereby, the
resulting process is quite limited in terms of control flow, since complex
behavior (such as the expressed by branching conditions, events,
synchronizations, etc.) cannot be defined.

Another important limitation of these approaches refers to process enactment.
The proposals by Ralyt\'e \cite{Ralyte01a,Ralyte01b} and Karlsson
\cite{Karlsson04,Karlsson09} do not consider process enactment. Prakash
\cite{Prakash97,Prakash99} defines an enactment algorithm that is oriented
towards the construction of specific method products and, therefore, does not
align with process enactment as we intend to support in our ME approach. The
support provided for process enactment in Brinkkemper's proposal
\cite{Brinkkemper96,Brinkkemper99} is more akin to our work. However, since only
two types of relationships (i.e., precedence and conditional precedence) can be
established between process fragments, the resulting processes, and
consequently their enactment, may be too limited for software engineers working
on real development projects.

Other important contribution to the ME field is the OPEN Process
Framework (OPF) \cite{Firesmith02}. The OPF provides a repository of method
modules (in OPF called method components) that are defined in terms of a
meta-model. This meta-model has recently been upgraded to fit the ISO/IEC 24744
standard \cite{iso24744}. While this standard does support the specification of
the process part of methods, it provides limited support for the definition of
processes with a complex control flow. In ISO/IEC 24744 process elements are
defined as \textit{work units}, which are allocated in \textit{stages}. The
control flow of the process is established by the stages, which can only be
associated via precedence relationships. Aharoni \textit{et al.} note this
problem in \cite{Aharoni08} and suggest enriching processes by means of an
extension of the stage concept that allows work units to be combined within
stages in more meaningful ways than simple inclusion (e.g., concurrently or
iteratively). Another important limitation refers to the lack of formalization
of process execution semantics, which hinders process enactment via a
process engine.

Another recent standard initiative is represented by the SPEM 2.0 standard
\cite{spem}. SPEM 2.0 defines a meta-model for development methods that
also presents the problems stated above regarding process (control flow)
specification and process enactment. However, SPEM 2.0 presents an important
advantage that helps overcome these problems, making it a suitable language
to be used in our proposal. Specifically, SPEM 2.0 provides powerful mechanisms for
enhancing process definitions via behavioral modeling formalisms such as BPMN
2.0 \cite{bpmn}. This allows method engineers to enhance the process
definition in terms of control flow specification and process
executability.

To sum up, we consider that the main process support limitations of ME
approaches are in terms process control flow specification and enactment. To
fill this gap, this paper extends the work presented by the authors in
\cite{Cervera10,Cervera11}. Our main intent is to meet ME needs regarding
support to the process part of methods.

\section{Overview of the MOSKitt4ME Approach}
\label{SectionOverview}

Figure \ref{FigureOverview} shows an overview of our ME approach. The proposal
is situated within the context of MDE. Following MDE principles, methods are
first defined as models (\textit{method design} phase) and these models are then
used by model transformations to generate the supporting CASE environments
(\textit{method implementation} phase). During the method design, the method
engineer defines both the product and process parts of the method. This is
carried out by assembling method fragments that are available in a method base
repository \cite{Cervera11}. Once the method design is finished, a CASE
environment that provides support to both the product and process parts of the
method is obtained during the method implementation. This is done by means of a
Model-To-Text (M2T) transformation.

\begin{figure}
\centering
\includegraphics[scale=0.55]{../images/Overview.png}
\caption{Overview of MOSKitt4ME}
\label{FigureOverview}
\end{figure}

This paper focuses on the process support provided in both phases of our ME
approach. The main goal of the work is to allow method engineers not only to
properly specify the process part of methods (during method design) but also to
bring this specification to execution (during method implementation). In order
to achieve this goal, our proposal has been defined based on a set of needs that
must be met. We consider these needs as a first step towards suitable process
support in ME and CAME technology.

With respect to method design, we need a language to properly specify the
process part of the method. This language must:

\begin{itemize}
  \item be expressive enough to enable the representation of
  complete, understandable, unambiguous, and well-structured processes. 
  
  \item fully formalize process execution semantics so that process
  enactment support can be provided in the CASE environment supporting the
  method.
\end{itemize}

With respect to method implementation, we need to enhance CASE environments
with mechanisms that provide support to:

\begin{itemize}
  \item the execution of the process specification. Process engines provide a
  set of enactment facilities (such as task orchestration, task automation,
  constraint enforcement, etc.) that allow CASE environments to provide
  guidance to software engineers throughout the actual development process and
  also to partially automate its performance.
  
  \item the management of the method products consumed or produced during
  the process execution. Process engines must be able to invoke the software
  tools that enable the creation and manipulation of the method products.
  Therefore, these tools (editors, generators, etc.) must be integrated in the
  CASE environment supporting the method.
\end{itemize}

In order to meet the needs regarding method design, our proposal combines the
use of SPEM 2.0 and BPMN 2.0 for various reasons. On the one hand, SPEM 2.0
provides better support than BPMN 2.0 for method modeling. On the other hand,
BPMN 2.0 provides more expressiveness than SPEM 2.0 with respect to process
specification. In addition, BPMN 2.0 fully formalizes execution semantics, while
SPEM 2.0 is not executable. Thereby, the method is defined in our proposal by
means of SPEM 2.0, and the process part is complemented with a BPMN 2.0 model
that enhances the process definition in terms of process (control flow)
specification and process executability. We provide a more detailed comparative
analysis of SPEM 2.0 and BPMN 2.0 in \cite{Cervera12}.

In order to meet the needs regarding method implementation, we have developed a
software component that is always integrated in the generated CASE environments.
This component is built upon a process engine that enables the execution of the
process definition by orchestrating method tasks and invoking tools for the
development of the method products. In order for the CASE environments to
contain the required tools, we make use of reusable assets. These assets are
stored in a repository and are associated to method elements during the method
design. Then, the M2T transformation takes these assets and integrates them in
the generated CASE environment.

\section{Process Specification during Method Design}
\label{SectionMethodDesign}

Figure \ref{FigureMethodDesign} shows our proposal for method design. It is
composed of three main steps: \textit{method definition}, \textit{method configuration}, and
\textit{executable process definition}. These steps are described below,
focusing on how the process specification is performed. Then, subsection
\ref{SubsectionExample} illustrates these steps with an example.

\begin{figure}
\centering
\includegraphics[width=\textwidth]{../images/MethodDesign.png}
\caption{Method design in MOSKitt4ME}
\label{FigureMethodDesign}
\end{figure}

\subsubsection{Method Definition.}

In this step, the method engineer builds the method model by means of SPEM
2.0. As figure \ref{FigureMethodDesign} illustrates, the main elements
that the method engineer must define to build the method process part
are \textit{Tasks}, \textit{Activities}, \textit{Work Sequences}, and
\textit{Roles}. Tasks represent basic units of work. Activities group tasks and
other activities, forming breakdown structures. The root activity of these
breakdown structures is named \textit{Delivery Process}. Work
sequences represent precedence relationships between tasks and activities. Roles
are performers of method tasks. On the other hand, the method product part is
composed of \textit{Work Products}, which are inputs and outputs of method
tasks.

The construction of this model can be performed by assembling reusable method
fragments retrieved from a method base repository. Further details about how
these fragments are managed in our ME approach can be found in
\cite{Cervera10,Cervera11}.

\subsubsection{Method Configuration.}

In this step\footnote{Note that this phase differs from the ME approach of
method configuration \cite{Karlsson04}.}, the method engineer defines the tools
that will allow software engineers to perform the method tasks during the
process enactment, and the guides that will assist them during the tasks
performance. To do so, the method engineer links tasks (and work products) with
\textit{reusable assets} that are retrieved from an asset base repository. These
assets can be \textit{tool assets}, which contain tools (editors, model
transformations, etc.) that enable the creation of products, and
\textit{guidance assets}, which contain guidelines (textual descriptions,
process models, etc.) about the performance of tasks.

The association of reusable assets with tasks and products is performed based on
the following observations: tasks can be associated with tool assets
containing model transformations. These transformations will be executed when
the tasks are invoked during the process enactment. These tasks are
considered automatic. Manual tasks are not associated with tool assets. Tasks
can also be associated with guidance assets. The guidelines contained in these
assets will be used as guidance for the user during the process enactment. On
the other hand, products must be associated with tool assets defining the
notation that will be used during the process enactment for the creation of the
products. These assets can contain meta-models or editors. A meta-model defines
the abstract syntax of a notation. An editor defines both the abstract and
concrete syntax.

To summarize, table \ref{tableAssociations} gathers the associations allowed
between method elements and reusable assets. Further details can be found in
\cite{Cervera10,Cervera11}.

\begin{table}
\caption{Associations allowed between method elements and reusable assets}
\centering
\begin{tabular}{| c | c |}
\cline{1-2}
\multicolumn{1}{|c|}{\textbf{Method Element Type}} &
\multicolumn{1}{|c|}{\textbf{Reusable Asset Type}}
\\
\hline
Task & Guidance Asset\\
\cline{2-2}
& Tool Asset (Model Transformation)\\
\hline
Work Product & Tool Asset (Meta-model)\\
\cline{2-2}
& Tool Asset (Editor)\\
\hline
\end{tabular}
\label{tableAssociations}
\end{table}

\subsubsection{Executable Process Definition.}

In this step, the method engineer defines an executable representation of the
method process part. We have automated this step by means of a Model-to-Model
(M2M) transformation that takes the configured SPEM 2.0 model as input and
automatically generates a set of BPMN 2.0 processes. Then, these processes can
be manually modified to complete the process specification\footnote{Ideally, the
M2M transformation will be implemented as a bidirectional transformation. Thus,
changes in the BPMN 2.0 model affecting the SPEM 2.0 model will be automatically
transmitted (and vice versa).}. This is often needed because BPMN 2.0 provides
more expresiveness than SPEM 2.0 with respect to process elicitation.

In order to provide more insights on how the transformation obtains the
BPMN 2.0 processes, we summarize in table \ref{tableMappings} the mappings
between the SPEM 2.0 and BPMN 2.0 concepts. In particular, table
\ref{tableMappings} contains mappings for all SPEM 2.0 concepts that belong to
the process part of a method (see figure \ref{FigureMethodDesign}). Logical
elements such as decision nodes are not included since they cannot be
represented in SPEM 2.0. The rationale of the mappings is provided below.

\begin{table}
\caption{Mappings between SPEM 2.0 and BPMN 2.0}
\centering
\begin{tabular}{| p{0.5cm} | p{5.5cm} | p{5.5cm} |}
\cline{2-3}
\multicolumn{1}{c|}{} & \multicolumn{1}{|c|}{\textbf{SPEM 2.0}} &
\multicolumn{1}{|c|}{\textbf{BPMN 2.0}}
\\
\hline
1 & DeliveryProcess (root Activity) & Process\\
\hline
2 & Activity (nested) & Process and CallActivity\\
\hline
3 & Task (with ReusableAsset) & ServiceTask\\
\hline
4 & Task (with ReusableAsset associated to output WorkProduct) &
UserTask\\
\hline
5 & Task (without ReusableAsset) & ManualTask\\
\hline
6 & WorkSequence & SequenceFlow\\
\hline
7 & Role & Lane \\
\hline
\end{tabular}
\label{tableMappings}
\end{table}

\begin{enumerate}
  \item A SPEM 2.0 \textit{DeliveryProcess} is mapped into a BPMN 2.0
  \textit{Process}.
  
  \item A SPEM 2.0 \textit{Activity} is mapped into a BPMN 2.0
  \textit{CallActivity} and a BPMN 2.0 \textit{Process}. The
  \textit{CallActivity} invokes the \textit{Process} when it is executed.
  
  \item A SPEM 2.0 \textit{Task} is mapped into a BPMN 2.0
  \textit{ServiceTask} if and only if a reusable asset containing a model
  transformation is associated to the \textit{Task}.
  
  \item A SPEM 2.0 \textit{Task} is mapped into a BPMN 2.0 \textit{UserTask}
  if and only if the \textit{Task} is not automatic (no reusable asset
  containing a model transformation is associated to it) and a reusable asset is
  associated to an output \textit{WorkProduct} of the \textit{Task}.
  
  \item A SPEM 2.0 \textit{Task} is mapped into a BPMN 2.0
  \textit{ManualTask} if and only if no reusable asset is associated to the
  \textit{Task} and its output \textit{WorkProducts}.
  
  \item A SPEM 2.0 \textit{WorkSequence} is mapped into a BPMN 2.0
  \textit{SequenceFlow}. The source of the \textit{SequenceFlow} is set to the
  BPMN 2.0 element generated from the \textit{predecessor} of the
  \textit{WorkSequence}. The target is set to the BPMN 2.0 element generated
  from the \textit{successor} of the \textit{WorkSequence}.
  
  \item A SPEM 2.0 \textit{Role} is mapped into a BPMN 2.0 \textit{Lane} if
  and only if the \textit{Lane} has not been previously generated.
\end{enumerate}

\subsection{An Example of Process Specification in MOSKitt4ME}
\label{SubsectionExample}  

In order to illustrate our proposal for process specification, we present a
practical example carried out in MOSKitt4ME. The example process has been
modeled in real settings at the CIT. Specifically, it corresponds to an excerpt
of gvMetrica that deals with the design of information systems.

\subsubsection{Method Definition.}

This step is performed via the EPF Composer \cite{epf}, an
Eclipse-based editor that has been integrated in MOSKitt4ME to enable the
creation of SPEM 2.0 models. Figure \ref{SPEMprocess} shows the
example process after being defined by means of this editor. As the figure
shows, the first steps of the process are to define the system architecture, the
user interface, and the business logic. Then, a model defining the system
database schema is obtained from the models specifying the business logic. The
database model is then used to automatically obtain the database code. Once all
the system artifacts have been created, the test cases can be defined. Finally,
the design validation is carried out to validate all the work performed during
the process.

As depicted in figure \ref{SPEMprocess}, the process is represented in SPEM 2.0
as a breakdown structure that is mainly composed of \textit{Activities} (e.g.,
\textit{Data Persistence Design}) and \textit{Tasks} (e.g.,
\textit{Database Model Generation}), which reference performing \textit{Roles}
(e.g., \textit{Analyst}) as well as input and output \textit{Work Products}
(e.g., \textit{UML Class Model}). Moreover, there are \textit{Work Sequences}
that are established between method elements (e.g., \textit{System Architecture
Definition} is a predecessor of \textit{User Interface Design}).

\begin{figure}
\centering
\includegraphics[scale=0.65]{../images/SPEMprocess.PNG}
\caption{Example process in SPEM 2.0}
\label{SPEMprocess}
\end{figure}

\subsubsection{Method Configuration.}

This step is performed via a repository client that allows method
engineers to retrieve reusable assets from the asset base \cite{Cervera11}. As
an example, let us consider the task \textit{Database Model Generation}. The
execution of this task obtains a database model from a UML class model. To
specify this behavior, the method engineer can associate the task with an asset
containing the Eclipse plug-ins that implement the UML2DB transformation
provided by the MOSKitt tool. On the other hand, the work products of the task
can be associated with assets containing the UML meta-model and the MOSKitt
SQLSchema meta-model respectively.

\subsubsection{Executable Process Definition.}

This step is performed via the M2M transformation described above, which has
been implemented in MOSKitt4ME as an extension of the EPF Composer. Figure
\ref{BPMNprocess} shows the BPMN 2.0 processes resulting from applying this
transformation to the example process. These processes are represented in terms
of the Activiti Designer \cite{activiti}, an Eclipse-based graphical editor that
has been integrated in MOSKitt4ME to support BPMN 2.0.

\begin{figure}
\centering
\includegraphics[scale=0.23]{../images/BPMNprocess.PNG}
\caption{Generated BPMN 2.0 processes}
\label{BPMNprocess}
\end{figure}

To illustrate how the processes shown in figure \ref{BPMNprocess} have
been generated, we present below some examples of the application of the
mappings of table \ref{tableMappings}.

\begin{enumerate}
  \item The SPEM 2.0 \textit{Delivery Process} ``\textit{Information System
  Design}'' is mapped into a BPMN 2.0 \textit{Process} (diagram
  ``\textit{A}'' in figure \ref{BPMNprocess}).
  
  \item The SPEM 2.0 \textit{Activity} ``\textit{Data Persistence Design}'' is
  mapped into a \textit{Call Activity} and a BPMN 2.0 \textit{Process}
  (diagram ``\textit{B}'' in figure \ref{BPMNprocess}).
  
  \item The SPEM 2.0 \textit{Task} ``\textit{Database Model Generation}'' is
  mapped into a \textit{Service Task} since it has a M2M transformation
  associated to it as a reusable asset.
  
  \item The SPEM 2.0 \textit{Task} ``\textit{Database Model Revision}'' is mapped
  into a \textit{User Task} since it is not automatic but has an output product with a
  reusable asset associated to it (not shown in the example).
  
  \item The SPEM 2.0 \textit{Task} ``\textit{Design Validation}'' is mapped into
  a \textit{Manual Task} since it does not have any reusable asset associated to
  it.
  
  \item The SPEM 2.0 \textit{Work Sequences} are mapped into the
  \textit{Sequence Flows} that connect the BPMN 2.0 elements in both diagrams.
\end{enumerate}

After the generation of the BPMN 2.0 processes, the method engineer can
manually modify them to specify more complex control flows. For instance, the
method engineer can add a \textit{Gateway} to specify that the \textit{Call
Activity} ``\textit{Tests Design}'' must not be executable until all its
predecessors are finished.

\section{Process Enactment during Method Implementation}
\label{SectionMethodImplementation}

The method implementation phase is in charge of the construction of the CASE
tool support for the method defined during the method design. In MOSKitt4ME,
CASE environments are semi-automatically built by means of a M2T transformation.
This transformation takes a method specification as input and obtains a software
environment that integrates the tools contained in the reusable assets (as well
as a set of static components). We detail this transformation in \cite{Cervera10b}.

In this section we focus on how CASE environments are structured to support
the process enactment. As figure \ref{FigureMethodImplementation} shows, CASE
environments are divided into three parts: components providing
\textit{method product support}, components providing \textit{method process
support}, and the \textit{Project Manager Component} (PMC). These parts are
introduced below. Subsection \ref{SubsectionExample2} illustrates them with an
example.

\begin{figure}
\centering
\includegraphics[scale=0.55]{../images/MethodImplementation.PNG}
\caption{Method implementation in MOSKitt4ME}
\label{FigureMethodImplementation}
\end{figure}

\subsubsection{Method Product Support.}

The software components that provide product support must enable the
creation of the method products. These components
correspond to the reusable assets associated to the method elements during the
method configuration. Specifically, \textit{Tool assets} allow software
engineers to create the method products by means of model transformations or
software applications such as graphical or textual editors. \textit{Guidance
assets} do not allow software engineers to create the method
products, but they do provide guidance on how the method tasks must be
performed to properly develop these products.

\subsubsection{Method Process Support.}

The software components that provide process support must enable the execution
of the BPMN 2.0 model. This functionality is provided by the Activiti Engine
\cite{activiti}. The behavior of the engine according to the different task
types is the following:

\begin{itemize}
  \item \emph{Service tasks}: when a service task becomes active, it is
  automatically executed. The execution of a service task invokes the model
  transformation that is associated to the task as a reusable asset.
  \item \emph{User tasks}: when a user task becomes active, the engine invokes
  the software tools that enable the creation of the output products of the
  task. These tools are associated to the products as reusable assets.
  \item \emph{Manual tasks}: when a manual task becomes active, the engine does
  not perform any action.
  \item \emph{Call activities}: when a call activity becomes active, the engine
  automatically starts a new instance of the BPMN 2.0 process referenced by the
  call activity.
\end{itemize}

\subsubsection{Project Manager Component.}

The PMC provides a graphical user interface for the
CASE environment and assists software engineers during the process enactment. To
achieve this goal, it makes use of the Activiti engine to execute BPMN 2.0
process instances and also makes use of the SPEM 2.0 model to extract
information about the method that is not represented in the BPMN 2.0 model. The
PMC is divided into the following Eclipse views:

\begin{itemize}
  \item \textit{Project Explorer}: This view is provided as
  part of the Eclipse platform and has been integrated in the PMC to show the
  projects that have been created and the resources they contain (files,
  folders, etc.). From this view, the software engineer can create new projects,
  delete existing projects, etc.
  
  \item \textit{Process}: This view shows the current state of the
  process instance associated to the project that is selected in the Project
  Explorer view. From this view, the software engineer can invoke the execution
  of the tasks that are executable. Once a task is finished, the PMC invokes the
  engine API to set the task as executed and proceed to the next state of the
  process. The Process view also enables task filtering based on the role of the
  users.
  
  \item \textit{Product Explorer}: This view shows a hierarchical picture of the
  artifacts that have been produced during the course of the project that is
  selected in the Project Explorer view. This hierarchy is based on domains,
  subdomains, and work product elements, which are read from the SPEM 2.0 model.
  The Product Explorer view also enables product filtering based on the user
  role.
  
  \item \textit{Help}: This view is provided as part of Eclipse and has been
  integrated in the PMC to provide guidance to software engineers during
  the performance of the method tasks. This view is dynamically updated based on
  the task that is selected in the Process view. The guides to show are known by
  the PMC because they were associated to the selected task as a reusable asset.
\end{itemize}

\subsection{An Example of Process Enactment in MOSKitt4ME}
\label{SubsectionExample2}

In order to illustrate how processes are enacted in MOSKitt4ME, we continue with
the example introduced in subsection \ref{SubsectionExample}. Let us consider
that a new project has been created by means of the Project Explorer view. When
this project is selected, the Process and Product Explorer views are updated
accordingly. At this point, the Process view shows the initial state of the
process and the Product Explorer view is empty. Now, let us consider that the
first three activities have been executed, and the next executable task is
\textit{Database Model Generation}. Since this task is automatic, the PMC
automatically invokes the associated transformation. Once the transformation is
finished, the task is set as executed and the process proceeds to the next
state. This state is illustrated in figure \ref{ProcessProductView}.

\begin{figure}
\centering
\includegraphics[scale=0.67]{../images/ProcessAndProductExplorerViews.PNG}
\caption{Process and Product Explorer views}
\label{ProcessProductView}
\end{figure}

The Process view (left)\footnote{Additional screenshots of the tool
can be found in our previous work \cite{Cervera10,Cervera11,Cervera10b} and also
in https://users.dsic.upv.es/$\sim$vtorres/moskitt4me/} shows the
tasks and activities in different colors depending on whether they have already
been executed (blue), are executable (green), or are not executable (red). Note
that, even though the Process view shows the process in terms of SPEM 2.0, the
process instance corresponds to an instance of a BPMN 2.0 process. This is
possible because there is a one-to-one correspondence between SPEM 2.0 tasks and
BPMN 2.0 tasks.

The Product Explorer view (right) depicts some artifacts that have been
produced during the process enactment. In this case, it is showing the input and
output products of the last executed task (i.e., \textit{Database Model
Generation}).

Now, let us consider that the user wants to proceed with the process execution.
To do this, the user selects the task \textit{Database Model Revision} in
the Process view. This action has a twofold effect. The Help
view is updated to show textual guidance about the task and, since the task is a
user task, the PMC opens the software tool that allows the software
engineer to carry out the task.

Once all the tasks have been executed, the process engine deletes the process
instance and, therefore, the project can be considered as concluded.

\section{Conclusions}
\label{SectionConclusions}

In this paper we present an extension of our ME approach and CAME environment
(MOSKitt4ME) \cite{Cervera10,Cervera11} so as to provide adequate support to
process specification during method design and process enactment during method
implementation. This extension builds on the idea of combining the use of SPEM
2.0 and BPMN 2.0. This is based on the fact that BPMN 2.0 can resolve SPEM 2.0
limitations with respect to process elicitation and process executability.

MOSKitt4ME is currently being used in real settings at the CIT. Since
practitioners at the CIT are familiar with the techniques, languages, and tools
used in our work (i.e., MDE, SPEM 2.0, BPMN 2.0, Eclipse, etc.), MOSKitt4ME
has had a low learning curve. This brings important benefits, such as shorter
periods of training, and lower overhead. The use of the tool in a real context
is also providing us initial feedback that allows us to identify limitations of
our work. For instance, MOSKitt4ME does not yet properly deal with the dynamic
nature of projects. Therefore, as future work we aim to provide support for
variability and evolution of methods (and CASE tools) at runtime.

\bibliographystyle{splncs03}
\bibliography{../references/er2012}

\end{document}
